\subsubsection*{\S 第二讲\ 软件过程的历史演变和经典工作}
\setcounter{problemname}{0}

\begin{problem}
	``Measure twice, cut once"描述的是下述哪个软件开发场景:
	\uline{B}    
    \vspace{-0.8em}
    \begin{multicols}{4}
        \begin{enumerate}[label=\Alph*.]
            \item 软件设计
            \item 代码评审
            \item 需求开发
            \item V\&V;
        \end{enumerate}
    \end{multicols}
    \vspace{-1em}
\end{problem}



\begin{problem}
	整体来看,我们可以把软件的发展分为三大阶段, 以下不属于三大主要阶段的是:
	\uline{C}    
    \vspace{-0.8em}
    \begin{multicols}{4}
        \begin{enumerate}[label=\Alph*.]
            \item 软硬件一体化
            \item 网络化和服务化
            \item 云计算化和云原生
            \item 软件成为独立产品
        \end{enumerate}
    \end{multicols}
    \vspace{-1em}
\end{problem}




\begin{problem}
	以下描述中,不属于软件开发本质困难或者本质挑战的是:
	\uline{A}    
    \vspace{-0.8em}
    \begin{multicols}{4}
        \begin{enumerate}[label=\Alph*.]
            \item 质量难题
            \item 复杂性
            \item 不可见性
            \item 一致性
        \end{enumerate}
    \end{multicols}
    \vspace{-1em}
\end{problem}




\begin{problem}
	以下描述中,哪一种实践是软硬件一体化阶段的典型实践:
	\uline{A}    
    \vspace{-0.8em}
    \begin{multicols}{4}
        \begin{enumerate}[label=\Alph*.]
            \item Code and Fix
            \item 迭代式开发
            \item 瀑布生命周期模型
            \item 成熟度模型
        \end{enumerate}
    \end{multicols}
    \vspace{-1em}
\end{problem}

