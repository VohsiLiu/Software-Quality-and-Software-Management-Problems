\subsubsection*{\S 第七讲\ 项目支持活动}
\setcounter{problemname}{0}

\begin{problem}
	下述产物中属于典型的配置项是:
	\uline{ABCD}    
    \vspace{-0.8em}
    \begin{multicols}{2}
        \begin{enumerate}[label=\Alph*.]
            \item 接口设计文档
            \item 源代码
            \item 用户手册
            \item 系统使用培训材料(视频、录像等)
        \end{enumerate}
    \end{multicols}
    \vspace{-1em}
\end{problem}



\begin{problem}
	团队内部的配置审计通常应该关注什么:
	\uline{ABCD}    
    \vspace{-0.8em}
    \begin{multicols}{4}
        \begin{enumerate}[label=\Alph*.]
            \item 物理审计
            \item 配置项列表
            \item 配置管理记录
            \item 基线计划
        \end{enumerate}
    \end{multicols}
    \vspace{-1em}
\end{problem}



\begin{problem}
	下列关于决策分析的论述中,不恰当的是:
	\uline{BD}    
        \begin{enumerate}[label=\Alph*.]
            \item 决策分析指南中最关键的是明确需要开展决策分析活动的判定标准,即什么场合之下需要开展正式的决策分析活动
            \item 评价方法是体现决策者利益诉求的关键,因此,需要谨慎设计
            \item 候选方案的识别应该晚于于评价标准
            \item 现实生活中的项目投标就是一个典型的决策分析活动
        \end{enumerate}
\end{problem}




\begin{problem}
	下列关于根因分析的论述中,不恰当的是: 
	\uline{AD}    
        \begin{enumerate}[label=\Alph*.]
            \item 根因分析必须基于丰富的数据来选择合适的问题
            \item 鱼骨图是根因分析的有效手段
            \item 典型地,可以从技术、人员、培训以及过程角度开展根因分析
            \item 根因分析活动终止的唯一特征就是找到相应的根因的明确解决方案
        \end{enumerate}
\end{problem}

