\documentclass[10pt,a4paper,UTF8]{ctexart}

\linespread{1.5}
\usepackage{geometry}%用于设置上下左右页边距
	\geometry{left=2.5cm,right=2.5cm,top=3.2cm,bottom=2.7cm}
\usepackage{xeCJK,amsmath,paralist,enumerate,booktabs,multirow,graphicx,subfig,setspace,listings,lastpage,hyperref}
\usepackage{amsthm, amssymb, bm, color, framed, graphicx, hyperref, mathrsfs}
\usepackage{mathrsfs}  
	\setlength{\parindent}{2em}
	\lstset{language=Matlab}%
\usepackage{fancyhdr}
\usepackage{graphicx}
\usepackage{subfloat}
\usepackage{listings}
\usepackage{xcolor}
\usepackage{float}
\usepackage{paralist}
\usepackage{setspace}
\usepackage{titlesec}
\usepackage{enumitem}
\usepackage{hyperref}
\usepackage{multirow}
\usepackage{threeparttable}
\usepackage{tcolorbox}
\usepackage{tabularx}
\usepackage{ulem}
\usepackage{longtable}
\usepackage{multicol}
\usepackage{pifont}
\usepackage{lipsum}
\usepackage{microtype}
\usepackage{wrapfig}
\usepackage[absolute,overlay]{textpos}
\usepackage{makecell}
\usepackage{soul}
\usepackage{xeCJKfntef}


\setmainfont{Times New Roman}[SmallCapsFont=TeX Gyre Termes:+smcp]

\hypersetup{
	colorlinks=true,
	linkcolor=black,
	urlcolor=black
}

\setenumerate{partopsep=0pt,topsep=0pt,itemsep=-2pt,leftmargin=2em}
\setitemize{itemsep=-2pt,partopsep=0pt,topsep=0pt,leftmargin=2em}

\titlespacing*{\section}{0pt}{3pt}{3pt}
\titlespacing*{\subsection}{0pt}{2pt}{2pt}
\titlespacing*{\subsubsection}{0pt}{1pt}{1pt}
\titlespacing*{\paragraph}{0pt}{0pt}{0pt}

\ctexset{secnumdepth=4,tocdepth=4}
\setlength{\parindent}{0pt}
\setstretch{1.35}


\setCJKmainfont[BoldFont={FZHei-B01},ItalicFont={FZKai-Z03}]{FZShuSong-Z01} 
\setCJKsansfont[BoldFont={FZHei-B01}]{FZKai-Z03} 
\setCJKmonofont[BoldFont={FZHei-B01}]{FZFangSong-Z02}
\setCJKfamilyfont{zhsong}{FZShuSong-Z01} 
\setCJKfamilyfont{zhhei}{FZHei-B01} 
\setCJKfamilyfont{zhkai}[BoldFont={FZHei-B01}]{FZKai-Z03} 
\setCJKfamilyfont{zhfs}[BoldFont={FZHei-B01}]{FZFangSong-Z02} 
\renewcommand*{\songti}{\CJKfamily{zhsong}} 
\renewcommand*{\heiti}{\CJKfamily{zhhei}} 
\renewcommand*{\kaishu}{\CJKfamily{zhkai}} 
\renewcommand*{\fangsong}{\CJKfamily{zhfs}}


\definecolor{mKeyword}{RGB}{0,0,255}          % bule
\definecolor{mString}{RGB}{160,32,240}        % purple
\definecolor{mComment}{RGB}{34,139,34}        % green
\definecolor{mNumber}{RGB}{128,128,128} 

\lstdefinestyle {njulisting} {
	basewidth = 0.5 em,
	lineskip = 3 pt,
	basicstyle = \small\ttfamily,
	% keywordstyle = \bfseries,
	commentstyle = \itshape\color{gray}, 
	basicstyle=\small\ttfamily,
	keywordstyle={\color{mKeyword}},     % sets color for keywords
	stringstyle={\color{mString}},       % sets color for strings
	commentstyle={\color{mComment}},     % sets color for comments
	numberstyle=\tiny\color{mNumber},
	numbers = left,
	captionpos = t,
	breaklines = true,
	xleftmargin = 1 em,
	xrightmargin = 0 em,
	frame=tlrb,
	tabsize=4
}

\lstset{
style = njulisting, % 调用上述样式 
flexiblecolumns % 允许调整字符宽度
}


%================= 基本格式预置 ===========================
\usepackage{fancyhdr}
\pagestyle{fancy}
\lhead{\textsc{Software Quality and Software Management}}
\rhead{软件质量与管理}
\cfoot{\thepage}
\renewcommand{\headrulewidth}{0.4pt}
\renewcommand{\theenumi}{(\arabic{enumi})}
\CTEXsetup[format={\bfseries\zihao{-3}}]{section}
\CTEXsetup[format={\bfseries\zihao{4}}]{subsection}
\CTEXsetup[format={\bfseries\zihao{-4}}]{subsubsection}


\renewcommand{\contentsname}{目录}  

%\definecolor{shadecolor}{RGB}{241, 241, 255}
\newcounter{problemname}
\newenvironment{problem}[1][]{
  \stepcounter{problemname}
  \par\noindent\textbf{\arabic{problemname}.\,}
  \ifx\relax#1\relax
  % 无参数时不显示方括号
  \else
    【#1】
  \fi
}{\vspace{0.25em}}
\newenvironment{solution}{\textbf{解答:}\kaishu }{\vspace{0.25em}}

\newcommand{\myline}{\uline{\ \ \ \ \ \ \ \ \ \ }}

\newcommand{\myunderline}[1]{\CJKunderline{\ #1 \ }}

\begin{document}
	\begin{center}
		\LARGE\textbf{软件质量与管理课上练习题目}
	\end{center}

	\setlength{\parskip}{0.25em}

	\subsubsection*{\S 第一讲\ 概述}
\setcounter{problemname}{0}

\begin{problem}
以下说法是否正确?为什么?
\begin{enumerate}[label=\arabic*.]
    \item 软件过程管理是软件项目管理应该要实现目标。
    \item “在公司导入敏捷过程是我们今年过程改进的主要目标。”
    \item XP与CMM/CMMI是对立的两种软件开发方法。
    \item CMM/CMMI不适合当今互联网环境的项目管理需求。
    \item PDCA和IDEAL不适合在敏捷环境中使用。
    \item 不同的软件开发过程应该使用不同的生命周期模型,反之亦如此。
\end{enumerate}
\end{problem}


\begin{problem}
	关于Brooks提及的软件开发本质难题,下列说法中不准确的是:
	%\uline{AB}    
        \begin{enumerate}[label=\Alph*.]
            \item 本质难题总共有四个,分别为复杂、不可见、可变和质量挑战
            \item 既然是本质难题,那就说明是根植于软件开发本身,因而是不可能在软件开发当中得到缓解
            \item 严格来说,只有不可见才是真正的“本质”难题,其他三个因项目而异
            \item 四大本质难题贯穿软件发展的不同历史段,但是在不同历史阶段,相互凸显程度不一样
        \end{enumerate}
\end{problem}




\begin{problem}
	下列软件应用和开发的典型特征中属于软硬件一体化阶段的是:
	%\uline{BC}    
    \vspace{-0.8em}
    \begin{multicols}{2}
        \begin{enumerate}[label=\Alph*.]
            \item 可以通过引入操作系统,摆脱了硬件束缚
            \item 几乎不需要考虑需求变更
            \item 缺乏科班的软件工程师
            \item 系统兼容对应软件开发的成败非常关键
        \end{enumerate}
    \end{multicols}
    \vspace{-1em}
\end{problem}




\begin{problem}
	下列哪些项不属于管理活动应该包含的要素?
	%\uline{ABD}    
    \vspace{-0.8em}
    \begin{multicols}{4}
        \begin{enumerate}[label=\Alph*.]
            \item 成本
            \item 质量
            \item 目标
            \item 工期
        \end{enumerate}
    \end{multicols}
    \vspace{-1em}
\end{problem}



\begin{problem}
	下列名词和术语中不属于软件过程的有哪些?
	%\uline{BD}    
    \vspace{-0.8em}
    \begin{multicols}{4}
        \begin{enumerate}[label=\Alph*.]
            \item SCRUM
            \item CMM/CMMI
            \item GATE方法
            \item IDEAL
        \end{enumerate}
    \end{multicols}
    \vspace{-1em}
\end{problem}




\begin{problem}
	CMM的创始人是哪位?
	%\uline{C}    
    \vspace{-0.8em}
    \begin{multicols}{4}
        \begin{enumerate}[label=\Alph*.]
            \item Boehm
            \item Juran
            \item Humphrey
            \item Crosby
        \end{enumerate}
    \end{multicols}
    \vspace{-1em}
\end{problem}



\begin{problem}
	XP规定开发人员每周工作时间不超过 \myline 小时,连续加班不可以超过两周,以免降低生产率。
	%\uline{B}    
    \vspace{-0.8em}
    \begin{multicols}{4}
        \begin{enumerate}[label=\Alph*.]
            \item 30
            \item 40
            \item 50
            \item 60
        \end{enumerate}
    \end{multicols}
    \vspace{-1em}
\end{problem}



\begin{problem}
	下列不属于看板方法典型实践的是?
	%\uline{BD}    
    \vspace{-0.8em}
    \begin{multicols}{4}
        \begin{enumerate}[label=\Alph*.]
            \item 可视化工作流
            \item 站立式会议
            \item 限定WIP
            \item 重构
        \end{enumerate}
    \end{multicols}
    \vspace{-1em}
\end{problem}


	\subsubsection*{\S 第二讲\ 软件过程的历史演变和经典工作}
\setcounter{problemname}{0}

\begin{problem}
	``Measure twice, cut once"描述的是下述哪个软件开发场景:
	%\uline{B}    
    \vspace{-0.8em}
    \begin{multicols}{4}
        \begin{enumerate}[label=\Alph*.]
            \item 软件设计
            \item 代码评审
            \item 需求开发
            \item V\&V;
        \end{enumerate}
    \end{multicols}
    \vspace{-1em}
\end{problem}



\begin{problem}
	整体来看,我们可以把软件的发展分为三大阶段, 以下不属于三大主要阶段的是:
	%\uline{C}    
    \vspace{-0.8em}
    \begin{multicols}{4}
        \begin{enumerate}[label=\Alph*.]
            \item 软硬件一体化
            \item 网络化和服务化
            \item 云计算化和云原生
            \item 软件成为独立产品
        \end{enumerate}
    \end{multicols}
    \vspace{-1em}
\end{problem}




\begin{problem}
	以下描述中,不属于软件开发本质困难或者本质挑战的是:
	%\uline{A}    
    \vspace{-0.8em}
    \begin{multicols}{4}
        \begin{enumerate}[label=\Alph*.]
            \item 质量难题
            \item 复杂性
            \item 不可见性
            \item 一致性
        \end{enumerate}
    \end{multicols}
    \vspace{-1em}
\end{problem}




\begin{problem}
	以下描述中,哪一种实践是软硬件一体化阶段的典型实践:
	%\uline{A}    
    \vspace{-0.8em}
    \begin{multicols}{4}
        \begin{enumerate}[label=\Alph*.]
            \item Code and Fix
            \item 迭代式开发
            \item 瀑布生命周期模型
            \item 成熟度模型
        \end{enumerate}
    \end{multicols}
    \vspace{-1em}
\end{problem}


	\subsubsection*{\S 第三讲\ 团队动力学}
\setcounter{problemname}{0}

\begin{problem}
	对比TSP和SCRUM,下列说法不恰当的是: 
	\uline{BC}    
        \begin{enumerate}[label=\Alph*.]
            \item 都是过程框架,需要填补具体实践之后才是一个可以工作的过程
            \item 一种是计划驱动方法,另外一种是敏捷方法
            \item SCRUM适合迭代式场景,TSP适合瀑布场景
            \item 两种方法都需要进行度量数据收集、分析,从而支持管理决策
        \end{enumerate}
\end{problem}




\begin{problem}
	以下特征适用麦克勒格Y理论(McGregors Theory Y)激励的场合是:
	\uline{D}    
        \begin{enumerate}[label=\Alph*.]
            \item 关注工作环境,薪金等
            \item 更喜欢经常的指导,避免承担责任,缺乏主动性
            \item 自我中心,对组织需求反应淡漠,反对变革
            \item 能够自我约束,自我导向与控制,渴望承担责任
        \end{enumerate}
\end{problem}



\begin{problem}
	以下关于马斯洛的需求层次理论描述不正确的是:
	\uline{D}    
        \begin{enumerate}[label=\Alph*.]
            \item 自我实现是寻求自尊(Esteem)
            \item 激励来自为没有满足的需求而努力奋斗
            \item 低层次的需求必须在高层次需求满足之前得到满足
            \item 满足高层次的需求的途径比满足低层次的途径更少
        \end{enumerate}
\end{problem}



\begin{problem}
	以下关于团队动力学的论述,不恰当的是: 
	\uline{A}    
        \begin{enumerate}[label=\Alph*.]
            \item 马斯洛的需求层次理论可以用来更好地维持激励水平
            \item 智力工作的激励方式中,应该尽可能使用鼓励承诺这种方式
            \item 麦克勒格的X理论适合用马斯洛底层需求激励
            \item 海兹伯格的激励理论区分为内在因素和外在因素两种
        \end{enumerate}
\end{problem}

\begin{solution}
A. 马斯洛的需求层次理论可用于指导激励手段的选择,不是激励维持手段
\end{solution}



	\subsubsection*{\S 第四讲\ 估算、计划和跟踪}
\setcounter{problemname}{0}

\begin{problem}
	下述关于WBS的描述中,哪些说法是不正确的? 
	%\uline{A}    
        \begin{enumerate}[label=\Alph*.]
            \item WBS应该对应OBS
            \item WBS提供了范围管理的基础
            \item WBS工作分解最底层的要素是实现目标的充分必要条件
            \item WBS分解的时候,同一层不能应用不同标准
        \end{enumerate}
\end{problem}




\begin{problem}
	下述关于EVM的描述中,哪些说法是不正确的?
	%\uline{B}    
    \vspace{-0.8em}
    \begin{multicols}{2}
        \begin{enumerate}[label=\Alph*.]
            \item EVM不适用于质量管理
            \item EVM的中级实现中引入成本信息
            \item EVM高度依赖估算准确
            \item EVM可以适应需求变更
        \end{enumerate}
    \end{multicols}
    \vspace{-1em}
\end{problem}



	\subsubsection*{\S 第五讲\ 质量管理}
\setcounter{problemname}{0}

\begin{problem}
	关于PSP质量管理策略,下列说法中正确的是:
	\uline{ABD}
        \begin{enumerate}[label=\Alph*.]
            \item 用缺陷管理替代质量管理,既有必要性,也有合理性
            \item 基本无缺陷的开发是通过开展高质量的评审来实现的
            \item 经过训练,评审是所有消除缺陷的手段当中最高效的
            \item PSP质量策略主要解决的是外部质量,而非内部质量
        \end{enumerate}
\end{problem}

\begin{solution}
C. 编译消除的效率高于评审的。D. 软件质量既有内部质量也有外部质量,外部质量面向最终用户,内部质量则不然,PSP 使用面向用户的视图。
\end{solution}




\begin{problem}
	关于DRL,下列说法中不正确的是:
	\uline{CD}
        \begin{enumerate}[label=\Alph*.]
            \item 这是一种模块级开发中质量控制的指标
            \item DRL以单元测试每小时发现缺陷率作为基准,考察上游其他缺陷消除阶段的消除效率
            \item DRL以单元测试发现的缺陷个数作为基准,考察上游其他缺陷消除阶段消除缺陷的效率
            \item DRL只能预测,不能度量
        \end{enumerate}
\end{problem}

\begin{solution}
C. 应该为每小时。D. DRL可以进行度量;虽然每小时注入多少不可知,但是每小时消除多少是可知的。
\end{solution}



\begin{problem}
	关于PQI,下列说法中不正确的是:
	\uline{BCD}    
    \vspace{-0.8em}
    \begin{multicols}{2}
        \begin{enumerate}[label=\Alph*.]
            \item PQI表征模块级别开发中的过程规范化程度
            \item PQI越高越好,可以充分保障质量
            \item PQI越低越好
            \item PQI不能用作质量规划
        \end{enumerate}
    \end{multicols}
    \vspace{-1em}
\end{problem}



\begin{problem}
	关于PQI,下列说法中正确的是:
	\uline{AB}
    \vspace{-0.8em}
    \begin{multicols}{2}
        \begin{enumerate}[label=\Alph*.]
            \item PQI可以辅助判断模块开发质量
            \item PQI可以提供过程改进的依据
            \item PQI确保大于1,从而确保开发质量
            \item PQI只能预测,不能度量
        \end{enumerate}
    \end{multicols}
    \vspace{-1em}
\end{problem}



\begin{problem}
	关于Yield,下列说法中正确的是: 
	\uline{ABCD}
    \vspace{-0.8em}
    \begin{multicols}{2}
        \begin{enumerate}[label=\Alph*.]
            \item Yield可以辅助判断模块开发质量
            \item Yield可以提供过程改进的依据
            \item Yield区分为Process Yield和Phase Yield
            \item Yield只能预测,不能度量
        \end{enumerate}
    \end{multicols}
    \vspace{-1em}
\end{problem}



\begin{problem}
	关于评审速度,下列说法中正确的是:
	\uline{C}
        \begin{enumerate}[label=\Alph*.]
            \item 进行代码评审的时候,控制评审速度不超过每小时1000LOC就能实现大部分质量要求
            \item 实战中,评审速度应该根据资源水平而定,时间充分就评审慢一些
            \item 文档评审速度应该控制每小时不超过4页
            \item 评审速度与人的技能有关,技能强的人可以突破每小时1000 LOC代码这个限制
        \end{enumerate}
\end{problem}




\begin{problem}
	关于Humphrey梳理的Quality Journey,下列说法中正确的是: 
	\uline{CD}
        \begin{enumerate}[label=\Alph*.]
            \item Quality Journey中列出的步骤可以在适当的时候更换顺序
            \item 由于需求是一切工程活动的基础,因此加强需求开发应该是Quality Journey早期的必备步骤
            \item Quality Journey仍然仅仅是在“用缺陷管理替代质量管理”这一基本策略之下进行讨论
            \item Quality Journey中测试应该先于评审得到贯彻和改善
        \end{enumerate}
\end{problem}



\begin{problem}
	下述设计模板中用来记录内部动态信息的是:
	\uline{B}
    \vspace{-0.8em}
    \begin{multicols}{4}
        \begin{enumerate}[label=\Alph*.]
            \item OST
            \item SST
            \item LST
            \item FST
        \end{enumerate}
    \end{multicols}
    \vspace{-1em}
\end{problem}



\begin{problem}
	下述关于PSP四大设计模板和UML典型设计图的描述中完全正确的是: 
	\uline{B}
        \begin{enumerate}[label=\Alph*.]
            \item OST在UML中没有对应的设计图
            \item UML中的类结构以及类之间的关系,在PSP四大设计模板中无法体现
            \item LST在UML中可以通过类图来体现
            \item FST在UML中可以通过类图来体现
        \end{enumerate}
\end{problem}

\begin{solution}
B. UML中的时序图和类图所描述的类之间的关系以及对象之间的交互信息在PSP4个设计模板中没有对应的内容。
\end{solution}



\begin{problem}
	一个完全正确的状态机应该满足:
	\uline{ABC}
    \vspace{-0.8em}
    \begin{multicols}{2}
        \begin{enumerate}[label=\Alph*.]
            \item 没有死循环和陷阱
            \item 状态转化条件满足正交性
            \item 状态转化条件满足完整性
            \item 状态转化条件满足独立性
        \end{enumerate}
    \end{multicols}
    \vspace{-1em}
\end{problem}




\begin{problem}
	下列关于各种设计验证手段的描述中正确的是:
	\uline{CD}
    \vspace{-0.8em}
    \begin{multicols}{2}
        \begin{enumerate}[label=\Alph*.]
            \item 执行表是唯一一种提供全面设计验证的手段
            \item 跟踪表是唯一一种提供全面设计验证的手段
            \item 受限于手工方式,都易于出错
            \item 符号化执行验证不适合复杂的计算过程
        \end{enumerate}
    \end{multicols}
    \vspace{-1em}
\end{problem}



\begin{problem}
	关于使用程序正确性证明手段验证while-do循环设计的描述中,正确的是:
	\uline{ABCD}
        \begin{enumerate}[label=\Alph*.]
            \item 如果设计是正确的,那么应满足的条件之一是循环判断条件最后一定可以变为false
            \item 如果设计是正确的,那么应满足的条件之一是循环判断条件为真的时候,单独的循环结构执行结果与循环体再加一个循环结构,其执行结果一致
            \item 如果设计是正确的,那么应满足的条件之一是循环判断条件为false的时候,循环体内所有变量不能被修改
            \item 该方法并不能保证循环体算法实现设计意图
        \end{enumerate}
\end{problem}



\begin{problem}
	下述设计验证手段的描述,哪些是正确的?
	\uline{A}
        \begin{enumerate}[label=\Alph*.]
            \item 符号化执行容易引入人为错误
            \item 状态机验证是唯一一种提供一般意义的上的正确性检验的验证手段
            \item 执行表的对设计缺陷的验证能力强于跟踪表
            \item 正确性检验是唯一可靠的设计验证手段
        \end{enumerate}
\end{problem}

	\subsubsection*{\S 第六讲\ 团队工程开发}
\setcounter{problemname}{0}

\begin{problem}
	下面描述属于典型客户需求的是: 
	%\uline{ABC}    
    \vspace{-0.8em}
    \begin{multicols}{4}
        \begin{enumerate}[label=\Alph*.]
            \item 客户期望
            \item 预算限制
            \item 法律法规限制
            \item 系统功能描述
        \end{enumerate}
    \end{multicols}
    \vspace{-1em}
\end{problem}



\begin{problem}
	典型地,在团队设计活动中,应该注意哪些内容:
	%\uline{ABCD}    
    \vspace{-0.8em}
    \begin{multicols}{4}
        \begin{enumerate}[label=\Alph*.]
            \item 设计标准的应用
            \item 复用的考虑
            \item 可测试性支持
            \item 可用性支持
        \end{enumerate}
    \end{multicols}
    \vspace{-1em}
\end{problem}




\begin{problem}
	关于集成策略,下述描述中正确的是: 
	%\uline{BCD}    
        \begin{enumerate}[label=\Alph*.]
            \item 当待集成组件质量普遍不高的时候,不可以使用扁平化策略
            \item 当需要尽早获取可以工作的组件的时候, 应该使用集簇式策略
            \item 当待集成组件质量普通较高的时候,可以使用大爆炸式集成策略
            \item 持续集成本质上就是逐一添加策略
        \end{enumerate}
\end{problem}




\begin{problem}
	当考虑集成策略的时候,应该注意如下哪些方面?
	%\uline{ABCD}      
    \vspace{-0.8em}
    \begin{multicols}{2}
        \begin{enumerate}[label=\Alph*.]
            \item 待集成组件的质量状态
            \item 待集成组件的获取方式
            \item 待集成组件的功能和关系
            \item 待集成组件的数量
        \end{enumerate}
    \end{multicols}
    \vspace{-1em}
\end{problem}



\begin{problem}
	关于扁平化集成策略和集簇式集成策略,下述说法中正确的是: 
	%\uline{BC}      
        \begin{enumerate}[label=\Alph*.]
            \item 扁平化策略可以较早地充分地暴露系统级别的错误
            \item 扁平化策略对于系统级别错误的暴露能力有限
            \item 集簇式集成策略有助于复用策略的实现
            \item 扁平化策略和集簇式策略的优缺点正好相反
        \end{enumerate}
\end{problem}



\begin{problem}
	在团队设计活动中,应该注意设计标准,下列属于典型的设计标准应该约定的是:
	%\uline{ABCD}    
    \vspace{-0.8em}
    \begin{multicols}{2}
        \begin{enumerate}[label=\Alph*.]
            \item 命名规范
            \item 接口标准
            \item 出错或者异常处理信息
            \item 设计表示方式
        \end{enumerate}
    \end{multicols}
    \vspace{-1em}
\end{problem}



\begin{problem}
	下述活动是典型的验证(Verification)的是:
	%\uline{ABC}    
    \vspace{-0.8em}
    \begin{multicols}{4}
        \begin{enumerate}[label=\Alph*.]
            \item 需求评审
            \item 详细设计评审
            \item 单元测试
            \item 试运行
        \end{enumerate}
    \end{multicols}
    \vspace{-1em}
\end{problem}



\begin{problem}
	下述活动是典型的确认(Validation)的是:
	%\uline{A}    
    \vspace{-0.8em}
    \begin{multicols}{4}
        \begin{enumerate}[label=\Alph*.]
            \item 验收测试
            \item 代码评审
            \item 系统测试
            \item 持续集成
        \end{enumerate}
    \end{multicols}
    \vspace{-1em}
\end{problem}



\begin{problem}
    下述产物中属于典型的确认(Validation)对象的是:
    %\uline{BCD}    
    \vspace{-0.8em}
    \begin{multicols}{2}
        \begin{enumerate}[label=\Alph*.]
            \item 接口设计文档
            \item 源代码
            \item 用户手册
            \item 系统使用培训材料(视频、录像等)
        \end{enumerate}
    \end{multicols}
    \vspace{-1em}
\end{problem}



\begin{problem}
	下述关于需求开发的描述中,哪些是正确的?
	%\uline{BC}    
    \vspace{-0.8em}
    \begin{multicols}{2}
        \begin{enumerate}[label=\Alph*.]
            \item 客户需求指客户的提出关于软件功能具体要求
            \item 工期或者预算往往都是客户需求的一个方面
            \item 产品需求需要跟客户充分讨论才能获取
            \item 客户应该在需求开发活动中起到主导作用
        \end{enumerate}
    \end{multicols}
    \vspace{-1em}
\end{problem}
	\subsubsection*{\S 第七讲\ 项目支持活动}
\setcounter{problemname}{0}

\begin{problem}
	下述产物中属于典型的配置项是:
	\uline{ABCD}    
    \vspace{-0.8em}
    \begin{multicols}{2}
        \begin{enumerate}[label=\Alph*.]
            \item 接口设计文档
            \item 源代码
            \item 用户手册
            \item 系统使用培训材料(视频、录像等)
        \end{enumerate}
    \end{multicols}
    \vspace{-1em}
\end{problem}



\begin{problem}
	团队内部的配置审计通常应该关注什么:
	\uline{ABCD}    
    \vspace{-0.8em}
    \begin{multicols}{4}
        \begin{enumerate}[label=\Alph*.]
            \item 物理审计
            \item 配置项列表
            \item 配置管理记录
            \item 基线计划
        \end{enumerate}
    \end{multicols}
    \vspace{-1em}
\end{problem}



\begin{problem}
	下列关于决策分析的论述中,不恰当的是:
	\uline{BD}    
        \begin{enumerate}[label=\Alph*.]
            \item 决策分析指南中最关键的是明确需要开展决策分析活动的判定标准,即什么场合之下需要开展正式的决策分析活动
            \item 评价方法是体现决策者利益诉求的关键,因此,需要谨慎设计
            \item 候选方案的识别应该晚于于评价标准
            \item 现实生活中的项目投标就是一个典型的决策分析活动
        \end{enumerate}
\end{problem}




\begin{problem}
	下列关于根因分析的论述中,不恰当的是: 
	\uline{AD}    
        \begin{enumerate}[label=\Alph*.]
            \item 根因分析必须基于丰富的数据来选择合适的问题
            \item 鱼骨图是根因分析的有效手段
            \item 典型地,可以从技术、人员、培训以及过程角度开展根因分析
            \item 根因分析活动终止的唯一特征就是找到相应的根因的明确解决方案
        \end{enumerate}
\end{problem}


	\subsubsection*{\S 其他题目}
\setcounter{problemname}{0}

\begin{problem}
	下列术语描述的技术或者方法是同类型的是?
	%\uline{CD}    
    \vspace{-0.8em}
    \begin{multicols}{2}
        \begin{enumerate}[label=\Alph*.]
            \item CMMI SPICE PDCA
            \item IDEAL XP SCRUM
            \item Cleanroom Gate TSP
            \item Waterfall SCRUM XP
        \end{enumerate}
    \end{multicols}
    \vspace{-1em}
\end{problem}




\begin{problem}
	在TSP的团队组建过程中,确定软件开发策略的是第几次会议?
	%\uline{C}    
    \vspace{-0.8em}
    \begin{multicols}{4}
        \begin{enumerate}[label=\Alph*.]
            \item 第一次
            \item 第二次
            \item 第三次
            \item 第四次
        \end{enumerate}
    \end{multicols}
    \vspace{-1em}
\end{problem}




\begin{problem}
	下列描述当中,属于过程经理的工作内容有哪些?
	%\uline{AC}    
    \vspace{-0.8em}
    \begin{multicols}{4}
        \begin{enumerate}[label=\Alph*.]
            \item 建立团队开发标准
            \item 主持项日周例会
            \item 记录周例会的记录
            \item 制定开发计划
        \end{enumerate}
    \end{multicols}
    \vspace{-1em}
\end{problem}



\begin{problem}
	下列关于挣值管理方法的描述中错误的是?
	%\uline{C}    
    \vspace{-0.8em}
    \begin{multicols}{2}
        \begin{enumerate}[label=\Alph*.]
            \item 这是一种可以用来跟踪项目预算消耗的方法
            \item 这种方法高度依赖估算准确性
            \item 这种方法可以支持质量管理
            \item 这种方法可以用来跟踪项目进度
        \end{enumerate}
    \end{multicols}
    \vspace{-1em}
\end{problem}



\begin{problem}
	完成一份完整的项目日程计划,需要下列哪些信息?
	%\uline{ABD}    
    \vspace{-0.8em}
    \begin{multicols}{4}
        \begin{enumerate}[label=\Alph*.]
            \item 任务清单
            \item 任务顺序
            \item 质量要求
            \item 人员资源水平
        \end{enumerate}
    \end{multicols}
    \vspace{-1em}
\end{problem}




\begin{problem}
	以下关于规模估算和度量的描述中,正确的是:
	%\uline{B}    
    \vspace{-0.8em}
    \begin{multicols}{2}
        \begin{enumerate}[label=\Alph*.]
            \item 功能点是一种可提供精确规模度量结果的方式
            \item 规模数据扮演了沟通历史数据的桥梁的角色
            \item 规模估算通常不用于质量计划当中
            \item PROBE 只用于规模估算
        \end{enumerate}
    \end{multicols}
    \vspace{-1em}
\end{problem}



\begin{problem}
	关于 PSP 缺陷日志,哪些信息是至关重要的:
	%\uline{AC}    
    \vspace{-0.8em}
    \begin{multicols}{2}
        \begin{enumerate}[label=\Alph*.]
            \item 缺陷发现时间
            \item 缺陷重现方式
            \item 缺陷根因描述
            \item 缺陷关联的其他缺陷
        \end{enumerate}
    \end{multicols}
    \vspace{-1em}
\end{problem}



\begin{problem}
	下列名词和术语中不属于软件过程的有哪些:
	%\uline{BD}    
    \vspace{-0.8em}
    \begin{multicols}{4}
        \begin{enumerate}[label=\Alph*.]
            \item SCRUM
            \item CMM/CMMI
            \item GATE 方法
            \item IDEAL
        \end{enumerate}
    \end{multicols}
    \vspace{-1em}
\end{problem}




\begin{problem}
	完成一份完整的项目日程计划,需要下列哪些信息:
	%\uline{ABD}    
    \vspace{-0.8em}
    \begin{multicols}{4}
        \begin{enumerate}[label=\Alph*.]
            \item 任务清单
            \item 任务顺序
            \item 质量要求
            \item 人员资源水平
        \end{enumerate}
    \end{multicols}
    \vspace{-1em}
\end{problem}



\begin{problem}
	下列术语描述的技术或者方法是同类型的是:
	%\uline{CD}    
    \vspace{-0.8em}
    \begin{multicols}{2}
        \begin{enumerate}[label=\Alph*.]
            \item CMMI SPICE PDCA
            \item IDEAL XP SCRUM
            \item Cleanroom Gate TSP
            \item Waterfall SCRUM XP
        \end{enumerate}
    \end{multicols}
    \vspace{-1em}
\end{problem}



\begin{problem}
	为了制定 Schedule plan,下述描述中,哪一项是不需要的:
	%\uline{A}    
    \vspace{-0.8em}
    \begin{multicols}{2}
        \begin{enumerate}[label=\Alph*.]
            \item Task size
            \item Task Order
            \item Schedule Hour
            \item Task hour for each task
        \end{enumerate}
    \end{multicols}
    \vspace{-1em}
\end{problem}




\begin{problem}
	在上题中,还需要补充下述哪一项数据就可以定义 Schedule Plan 了:
	%\uline{A}    
    \vspace{-0.8em}
    \begin{multicols}{4}
        \begin{enumerate}[label=\Alph*.]
            \item Task List
            \item Plan Value
            \item Earned Value
            \item Nothing
        \end{enumerate}
    \end{multicols}
    \vspace{-1em}
\end{problem}




\end{document}

