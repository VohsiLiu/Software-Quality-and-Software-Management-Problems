\subsubsection*{\S 第三讲\ 团队动力学}
\setcounter{problemname}{0}

\begin{problem}
	对比TSP和SCRUM,下列说法不恰当的是: 
	%\uline{BC}    
        \begin{enumerate}[label=\Alph*.]
            \item 都是过程框架,需要填补具体实践之后才是一个可以工作的过程
            \item 一种是计划驱动方法,另外一种是敏捷方法
            \item SCRUM适合迭代式场景,TSP适合瀑布场景
            \item 两种方法都需要进行度量数据收集、分析,从而支持管理决策
        \end{enumerate}
\end{problem}




\begin{problem}
	以下特征适用麦克勒格Y理论(McGregors Theory Y)激励的场合是:
	%\uline{D}    
        \begin{enumerate}[label=\Alph*.]
            \item 关注工作环境,薪金等
            \item 更喜欢经常的指导,避免承担责任,缺乏主动性
            \item 自我中心,对组织需求反应淡漠,反对变革
            \item 能够自我约束,自我导向与控制,渴望承担责任
        \end{enumerate}
\end{problem}



\begin{problem}
	以下关于马斯洛的需求层次理论描述不正确的是:
	%\uline{D}    
        \begin{enumerate}[label=\Alph*.]
            \item 自我实现是寻求自尊(Esteem)
            \item 激励来自为没有满足的需求而努力奋斗
            \item 低层次的需求必须在高层次需求满足之前得到满足
            \item 满足高层次的需求的途径比满足低层次的途径更少
        \end{enumerate}
\end{problem}



\begin{problem}
	以下关于团队动力学的论述,不恰当的是: 
	%\uline{A}    
        \begin{enumerate}[label=\Alph*.]
            \item 马斯洛的需求层次理论可以用来更好地维持激励水平
            \item 智力工作的激励方式中,应该尽可能使用鼓励承诺这种方式
            \item 麦克勒格的X理论适合用马斯洛底层需求激励
            \item 海兹伯格的激励理论区分为内在因素和外在因素两种
        \end{enumerate}
\end{problem}



