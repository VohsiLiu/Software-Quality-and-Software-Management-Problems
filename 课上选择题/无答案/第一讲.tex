\subsubsection*{\S 第一讲\ 概述}
\setcounter{problemname}{0}

\begin{problem}
以下说法是否正确?为什么?
\begin{enumerate}[label=\arabic*.]
    \item 软件过程管理是软件项目管理应该要实现目标。
    \item “在公司导入敏捷过程是我们今年过程改进的主要目标。”
    \item XP与CMM/CMMI是对立的两种软件开发方法。
    \item CMM/CMMI不适合当今互联网环境的项目管理需求。
    \item PDCA和IDEAL不适合在敏捷环境中使用。
    \item 不同的软件开发过程应该使用不同的生命周期模型,反之亦如此。
\end{enumerate}
\end{problem}


\begin{problem}
	关于Brooks提及的软件开发本质难题,下列说法中不准确的是:
	%\uline{AB}    
        \begin{enumerate}[label=\Alph*.]
            \item 本质难题总共有四个,分别为复杂、不可见、可变和质量挑战
            \item 既然是本质难题,那就说明是根植于软件开发本身,因而是不可能在软件开发当中得到缓解
            \item 严格来说,只有不可见才是真正的“本质”难题,其他三个因项目而异
            \item 四大本质难题贯穿软件发展的不同历史段,但是在不同历史阶段,相互凸显程度不一样
        \end{enumerate}
\end{problem}




\begin{problem}
	下列软件应用和开发的典型特征中属于软硬件一体化阶段的是:
	%\uline{BC}    
    \vspace{-0.8em}
    \begin{multicols}{2}
        \begin{enumerate}[label=\Alph*.]
            \item 可以通过引入操作系统,摆脱了硬件束缚
            \item 几乎不需要考虑需求变更
            \item 缺乏科班的软件工程师
            \item 系统兼容对应软件开发的成败非常关键
        \end{enumerate}
    \end{multicols}
    \vspace{-1em}
\end{problem}




\begin{problem}
	下列哪些项不属于管理活动应该包含的要素?
	%\uline{ABD}    
    \vspace{-0.8em}
    \begin{multicols}{4}
        \begin{enumerate}[label=\Alph*.]
            \item 成本
            \item 质量
            \item 目标
            \item 工期
        \end{enumerate}
    \end{multicols}
    \vspace{-1em}
\end{problem}



\begin{problem}
	下列名词和术语中不属于软件过程的有哪些?
	%\uline{BD}    
    \vspace{-0.8em}
    \begin{multicols}{4}
        \begin{enumerate}[label=\Alph*.]
            \item SCRUM
            \item CMM/CMMI
            \item GATE方法
            \item IDEAL
        \end{enumerate}
    \end{multicols}
    \vspace{-1em}
\end{problem}




\begin{problem}
	CMM的创始人是哪位?
	%\uline{C}    
    \vspace{-0.8em}
    \begin{multicols}{4}
        \begin{enumerate}[label=\Alph*.]
            \item Boehm
            \item Juran
            \item Humphrey
            \item Crosby
        \end{enumerate}
    \end{multicols}
    \vspace{-1em}
\end{problem}



\begin{problem}
	XP规定开发人员每周工作时间不超过 \myline 小时,连续加班不可以超过两周,以免降低生产率。
	%\uline{B}    
    \vspace{-0.8em}
    \begin{multicols}{4}
        \begin{enumerate}[label=\Alph*.]
            \item 30
            \item 40
            \item 50
            \item 60
        \end{enumerate}
    \end{multicols}
    \vspace{-1em}
\end{problem}



\begin{problem}
	下列不属于看板方法典型实践的是?
	%\uline{BD}    
    \vspace{-0.8em}
    \begin{multicols}{4}
        \begin{enumerate}[label=\Alph*.]
            \item 可视化工作流
            \item 站立式会议
            \item 限定WIP
            \item 重构
        \end{enumerate}
    \end{multicols}
    \vspace{-1em}
\end{problem}

