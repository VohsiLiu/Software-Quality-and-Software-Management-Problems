\subsubsection*{\S 第六讲\ 团队工程开发}
\setcounter{problemname}{0}

\begin{problem}
	下面描述属于典型客户需求的是: 
	%\uline{ABC}    
    \vspace{-0.8em}
    \begin{multicols}{4}
        \begin{enumerate}[label=\Alph*.]
            \item 客户期望
            \item 预算限制
            \item 法律法规限制
            \item 系统功能描述
        \end{enumerate}
    \end{multicols}
    \vspace{-1em}
\end{problem}



\begin{problem}
	典型地,在团队设计活动中,应该注意哪些内容:
	%\uline{ABCD}    
    \vspace{-0.8em}
    \begin{multicols}{4}
        \begin{enumerate}[label=\Alph*.]
            \item 设计标准的应用
            \item 复用的考虑
            \item 可测试性支持
            \item 可用性支持
        \end{enumerate}
    \end{multicols}
    \vspace{-1em}
\end{problem}




\begin{problem}
	关于集成策略,下述描述中正确的是: 
	%\uline{BCD}    
        \begin{enumerate}[label=\Alph*.]
            \item 当待集成组件质量普遍不高的时候,不可以使用扁平化策略
            \item 当需要尽早获取可以工作的组件的时候, 应该使用集簇式策略
            \item 当待集成组件质量普通较高的时候,可以使用大爆炸式集成策略
            \item 持续集成本质上就是逐一添加策略
        \end{enumerate}
\end{problem}




\begin{problem}
	当考虑集成策略的时候,应该注意如下哪些方面?
	%\uline{ABCD}      
    \vspace{-0.8em}
    \begin{multicols}{2}
        \begin{enumerate}[label=\Alph*.]
            \item 待集成组件的质量状态
            \item 待集成组件的获取方式
            \item 待集成组件的功能和关系
            \item 待集成组件的数量
        \end{enumerate}
    \end{multicols}
    \vspace{-1em}
\end{problem}



\begin{problem}
	关于扁平化集成策略和集簇式集成策略,下述说法中正确的是: 
	%\uline{BC}      
        \begin{enumerate}[label=\Alph*.]
            \item 扁平化策略可以较早地充分地暴露系统级别的错误
            \item 扁平化策略对于系统级别错误的暴露能力有限
            \item 集簇式集成策略有助于复用策略的实现
            \item 扁平化策略和集簇式策略的优缺点正好相反
        \end{enumerate}
\end{problem}



\begin{problem}
	在团队设计活动中,应该注意设计标准,下列属于典型的设计标准应该约定的是:
	%\uline{ABCD}    
    \vspace{-0.8em}
    \begin{multicols}{2}
        \begin{enumerate}[label=\Alph*.]
            \item 命名规范
            \item 接口标准
            \item 出错或者异常处理信息
            \item 设计表示方式
        \end{enumerate}
    \end{multicols}
    \vspace{-1em}
\end{problem}



\begin{problem}
	下述活动是典型的验证(Verification)的是:
	%\uline{ABC}    
    \vspace{-0.8em}
    \begin{multicols}{4}
        \begin{enumerate}[label=\Alph*.]
            \item 需求评审
            \item 详细设计评审
            \item 单元测试
            \item 试运行
        \end{enumerate}
    \end{multicols}
    \vspace{-1em}
\end{problem}



\begin{problem}
	下述活动是典型的确认(Validation)的是:
	%\uline{A}    
    \vspace{-0.8em}
    \begin{multicols}{4}
        \begin{enumerate}[label=\Alph*.]
            \item 验收测试
            \item 代码评审
            \item 系统测试
            \item 持续集成
        \end{enumerate}
    \end{multicols}
    \vspace{-1em}
\end{problem}



\begin{problem}
    下述产物中属于典型的确认(Validation)对象的是:
    %\uline{BCD}    
    \vspace{-0.8em}
    \begin{multicols}{2}
        \begin{enumerate}[label=\Alph*.]
            \item 接口设计文档
            \item 源代码
            \item 用户手册
            \item 系统使用培训材料(视频、录像等)
        \end{enumerate}
    \end{multicols}
    \vspace{-1em}
\end{problem}



\begin{problem}
	下述关于需求开发的描述中,哪些是正确的?
	%\uline{BC}    
    \vspace{-0.8em}
    \begin{multicols}{2}
        \begin{enumerate}[label=\Alph*.]
            \item 客户需求指客户的提出关于软件功能具体要求
            \item 工期或者预算往往都是客户需求的一个方面
            \item 产品需求需要跟客户充分讨论才能获取
            \item 客户应该在需求开发活动中起到主导作用
        \end{enumerate}
    \end{multicols}
    \vspace{-1em}
\end{problem}