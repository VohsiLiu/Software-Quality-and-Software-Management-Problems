\subsubsection*{\S 其他题目}
\setcounter{problemname}{0}

\begin{problem}
	下列术语描述的技术或者方法是同类型的是?
	%\uline{CD}    
    \vspace{-0.8em}
    \begin{multicols}{2}
        \begin{enumerate}[label=\Alph*.]
            \item CMMI SPICE PDCA
            \item IDEAL XP SCRUM
            \item Cleanroom Gate TSP
            \item Waterfall SCRUM XP
        \end{enumerate}
    \end{multicols}
    \vspace{-1em}
\end{problem}




\begin{problem}
	在TSP的团队组建过程中,确定软件开发策略的是第几次会议?
	%\uline{C}    
    \vspace{-0.8em}
    \begin{multicols}{4}
        \begin{enumerate}[label=\Alph*.]
            \item 第一次
            \item 第二次
            \item 第三次
            \item 第四次
        \end{enumerate}
    \end{multicols}
    \vspace{-1em}
\end{problem}




\begin{problem}
	下列描述当中,属于过程经理的工作内容有哪些?
	%\uline{AC}    
    \vspace{-0.8em}
    \begin{multicols}{4}
        \begin{enumerate}[label=\Alph*.]
            \item 建立团队开发标准
            \item 主持项日周例会
            \item 记录周例会的记录
            \item 制定开发计划
        \end{enumerate}
    \end{multicols}
    \vspace{-1em}
\end{problem}



\begin{problem}
	下列关于挣值管理方法的描述中错误的是?
	%\uline{C}    
    \vspace{-0.8em}
    \begin{multicols}{2}
        \begin{enumerate}[label=\Alph*.]
            \item 这是一种可以用来跟踪项目预算消耗的方法
            \item 这种方法高度依赖估算准确性
            \item 这种方法可以支持质量管理
            \item 这种方法可以用来跟踪项目进度
        \end{enumerate}
    \end{multicols}
    \vspace{-1em}
\end{problem}



\begin{problem}
	完成一份完整的项目日程计划,需要下列哪些信息?
	%\uline{ABD}    
    \vspace{-0.8em}
    \begin{multicols}{4}
        \begin{enumerate}[label=\Alph*.]
            \item 任务清单
            \item 任务顺序
            \item 质量要求
            \item 人员资源水平
        \end{enumerate}
    \end{multicols}
    \vspace{-1em}
\end{problem}




\begin{problem}
	以下关于规模估算和度量的描述中,正确的是:
	%\uline{B}    
    \vspace{-0.8em}
    \begin{multicols}{2}
        \begin{enumerate}[label=\Alph*.]
            \item 功能点是一种可提供精确规模度量结果的方式
            \item 规模数据扮演了沟通历史数据的桥梁的角色
            \item 规模估算通常不用于质量计划当中
            \item PROBE 只用于规模估算
        \end{enumerate}
    \end{multicols}
    \vspace{-1em}
\end{problem}



\begin{problem}
	关于 PSP 缺陷日志,哪些信息是至关重要的:
	%\uline{AC}    
    \vspace{-0.8em}
    \begin{multicols}{2}
        \begin{enumerate}[label=\Alph*.]
            \item 缺陷发现时间
            \item 缺陷重现方式
            \item 缺陷根因描述
            \item 缺陷关联的其他缺陷
        \end{enumerate}
    \end{multicols}
    \vspace{-1em}
\end{problem}



\begin{problem}
	下列名词和术语中不属于软件过程的有哪些:
	%\uline{BD}    
    \vspace{-0.8em}
    \begin{multicols}{4}
        \begin{enumerate}[label=\Alph*.]
            \item SCRUM
            \item CMM/CMMI
            \item GATE 方法
            \item IDEAL
        \end{enumerate}
    \end{multicols}
    \vspace{-1em}
\end{problem}




\begin{problem}
	完成一份完整的项目日程计划,需要下列哪些信息:
	%\uline{ABD}    
    \vspace{-0.8em}
    \begin{multicols}{4}
        \begin{enumerate}[label=\Alph*.]
            \item 任务清单
            \item 任务顺序
            \item 质量要求
            \item 人员资源水平
        \end{enumerate}
    \end{multicols}
    \vspace{-1em}
\end{problem}



\begin{problem}
	下列术语描述的技术或者方法是同类型的是:
	%\uline{CD}    
    \vspace{-0.8em}
    \begin{multicols}{2}
        \begin{enumerate}[label=\Alph*.]
            \item CMMI SPICE PDCA
            \item IDEAL XP SCRUM
            \item Cleanroom Gate TSP
            \item Waterfall SCRUM XP
        \end{enumerate}
    \end{multicols}
    \vspace{-1em}
\end{problem}



\begin{problem}
	为了制定 Schedule plan,下述描述中,哪一项是不需要的:
	%\uline{A}    
    \vspace{-0.8em}
    \begin{multicols}{2}
        \begin{enumerate}[label=\Alph*.]
            \item Task size
            \item Task Order
            \item Schedule Hour
            \item Task hour for each task
        \end{enumerate}
    \end{multicols}
    \vspace{-1em}
\end{problem}




\begin{problem}
	在上题中,还需要补充下述哪一项数据就可以定义 Schedule Plan 了:
	%\uline{A}    
    \vspace{-0.8em}
    \begin{multicols}{4}
        \begin{enumerate}[label=\Alph*.]
            \item Task List
            \item Plan Value
            \item Earned Value
            \item Nothing
        \end{enumerate}
    \end{multicols}
    \vspace{-1em}
\end{problem}


